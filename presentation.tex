\documentclass[xcolor=x11names, compress, handout]{beamer}
% \documentclass[xcolor=x11names, compress]{beamer}
\usepackage{sty/packages}
\usepackage{sty/commands}
\usepackage{sty/beamer_style}



\AtBeginSection[]{
  \AtBeginSection[]
  {
     \begin{frame}[noframenumbering, plain]

  \frametitle{Outline}
  %\hfill                        
  \centering
  \begin{minipage}[t][0.5\textheight]{0.75\textwidth}
   \linespread{2.0}
   \tableofcontents[currentsection]
 \end{minipage}
     \end{frame}
  }
}

%THIS IS THE INFO AT THE BOTTOM OF YOUR FRAMES
\title{BART}
\author{J.S. Rehak}
\date{June \nth{3}, 2020}

%%%%%%%%%%%%%%%%%%%%%%%%%%%%%%%%%%%%%%%%%%%%%%%%%%

\begin{document}

%%%%%%%%%%%%%%%%%%%%%%%%%%%%%%%%%%%%%%%%%%%%%%%%%%%%%%
\begin{frame}[plain]

\title{Assessing the Effectiveness of Acceleration Methods for
  Deterministic Neutron Transport Solvers \\ \large{Building a new tool for developers.}}

\author{
\begin{tabular}{c}
J. S. Rehak\\
\vspace{10pt}\\
\includegraphics[height=2.5cm]{0bk.eps}
\end{tabular}}
\date{\vspace{-20pt}\\
  \begin{tabular}{c}
    \large{ANS Summer Meeting: Acceleration Methods} \\
 June \nth{3}, 2020
  \end{tabular}
}

\titlepage
\end{frame}

\begin{frame}
  \frametitle{Outline}
  \centering
  \begin{minipage}[t][0.5\textheight]{0.75\textwidth}
   %\linespread{2.0}
   \tableofcontents
 \end{minipage}
\end{frame}

% Introduction (2 min)
\section{Introduction}
% - SS Boltzman Equation
\begin{frame}
  \frametitle{Steady-state Boltzman Transport Equation}
  Our problem of interest is the time-independent transport equation
  on a domain of interest $\rvec \in V$~\cite{lewis1993},
  
\begin{align*}
  &\left[ \oh\cdot  \nabla + \Sigma_t(\rvec,E) \right] \psi(\rvec,E,\oh)\\
  & \quad \quad \quad= \int_0^{\infty}dE'\int_{4\pi}d\oh'\Sigma_s(\rvec, E'\rightarrow E,\oh'\rightarrow\oh)
    \psi(\rvec,E',\oh') \\ & \quad \quad \quad+ Q(\rvec, E, \oh)\;,
\end{align*}
with a given boundary condition,
\begin{align*}
  \psi(\rvec,E,\oh) = \Gamma(\rvec, E, \oh), \quad \rvec \in \partial V,
  \quad \oh \cdot \hat{n} < 0
\end{align*}  
\end{frame}
% - Iterative methods
% -- Power iteration: dominance ratio (NDA)
% -- Scattering-source iteration: high scattering ratio (DSA)
% Assessing acceleration methods (5 min)
% - Definition of acceleration (3 min)
% -- Error removal per unit work
% -- Computational work definition: inversions
% -- When this work assumption breaks down
% - Race car analogy
% -- First past-the-post assessment
% -- Need a standard test track with a standard car that is accessible
% - Motivation for a new tool -- test track
% BART Code Goals (10 min)
% - Design goals
% -- Reliving some development burden
% -- Provide a controlled environment
% -- Provide instrumentation
% - Polymorphism
% - Built-in instrumentation
% -- Fourier analysis (possibility)
% - Benchmarks provided with code
% State of the code (2 min)
% - Finite element (1 min)
% -- Dealii
% -- Second order forms (SAAF, Diffusion)
% -- NDA being implemented
% - Code details (1 min)
% -- C++17 (Planned C++20)
% -- GoogleTest/GoogleMock
% -- Coverage and CI
% -- Protocol Buffers
% Future work (1 min)
% - More acceleration methods
% - Combinations of acceleration methods

\section{Section 2}
\frame[c]{\frametitle{Other stuff}

}

\section{References}

\frame[noframenumbering, plain]{\frametitle{References}
\tiny{\bibliographystyle{plain}}
\bibliography{bib/bib}
}

\appendix

\section{Backup Slides}

\frame[c]{\frametitle{Backup Slide}
  Backup
  }

\end{document}

%%% Local Variables:
%%% mode: latex
%%% TeX-master: t
%%% End:
